\documentclass[12pt,a4paper]{article}
\usepackage[utf8]{inputenc}
\usepackage[russian]{babel}
\usepackage[OT1]{fontenc}
\usepackage{amsmath}
\usepackage{amsfonts}
\usepackage{amssymb}
\begin{document}
\begin{center}\LARGE
Task Solver~--- средство решения математических задач с подробным решением.
\end{center}

\section{Средства реализации идеи}

Основными средствами создания программы будут выступать язык программирования \textbf{LISP} и основанный на нем пакет символьных математических вычислений \textbf{Maxima}. 

Так как программа на языке \textbf{Maxima} умеет переключаться на \textbf{LISP} и обратно, а вот \textbf{LISP} в \textbf{Maxima} я не знаю как переключить, то естественно будет создавать изначально batch-файлы для \textbf{Maxima}, в которых использовать код \textbf{LISP}.

Функции программы будут разделены на поисковую систему (retrieval system), текстовую систему (text system) и решательную систему (solving system).

Основа решательной системы~--- формальные задачи (formal task), реализованные в виде функций...

Зависимость модулей

\section{История работы программы}

В результате работы программы предполагается вести несколько логов:

Статистика (\textbf{./service\_files/statistic}).

Этот файл будет вестись в формате максимы и подгружаться точно также как обычные модули, но, в отличие от остальных модулей, он будет перезаписываться после каждого завершения рабочей сессии. Содержит две переменные:
\begin{itemize}
\item \textbf{function\_calls\_statistical\_data} --- для статистики вызовов функций
\item \textbf{project\_processing\_statistical\_data} --- для статистики обработки проектов
\end{itemize}


В функцию \textbf{debuglog} передаются следующие параметры: строка сообщения, параметры сообщения.

\section{Основные структуры данных}

Структура переменной типа <<проект>>~--- список, содержащий следующие поля:
\begin{itemize}
\item \textbf{entity\_type}~--- идентификатор типа, должно быть \textbf{tsproject}
\item \textbf{name}~--- имя проекта, по этому имени будут названы все соответствующие файлы проекта, а также заголовок документа
\item \textbf{creation\_time}~--- время создания проекта
\item \textbf{compil\_time}~--- время его последней попытки компиляции
\item \textbf{compiled\_file}~--- имя скомпилированного файла
\item \textbf{ts\_tasks}~--- список заданий типа \textbf{tstask}
\end{itemize}

Структура переменной типа <<задание>>~--- список, содержащий поля:
\begin{itemize}
\item \textbf{entity\_type}~--- идентификатор типа, должно быть \textbf{tstask}
\item \textbf{name}~--- имя задания, будет написано как подзаголовок перед решением
\item \textbf{informal\_task}~--- соответствующее неформальное задание
\item \textbf{parameters}~--- список параметров для решения
\end{itemize}

\section{Генерация контрольных и тестов для решения}

Для генерации теста или контрольной работы нужно передать следующие параметры:
\begin{itemize}
\item Нужны ли варианты ответов (т.\,е. контрольная это или тест)
\item Количество вариантов теста
\item Количество вариантов ответов для каждого задания
\item Варианты ответов генерируются для каждого задания или для всех сразу
\item Задания для теста
\item Нужна ли таблица для записи ответов
\end{itemize}

\end{document}