Информация справедлива для версии 0.0.191

\subsection{Найти скалярное произведение двух векторов}

\noindent \textbf{Команда для почты:} \mbox{ mg\_vsp }

\noindent \textbf{Текст задания:} Найти скалярное произведение векторов $\vec{v}$ и $\vec{w}$

\noindent \textbf{Аргументы команды:}\\ $v$ --- последовательность значений, первое из которых есть целое число, определяющее количество следующих за ним значений; каждое это  значение должно быть действительным числом;\\
$w$ --- последовательность значений, первое из которых есть целое число, определяющее количество следующих за ним значений; каждое это  значение должно быть действительным числом;\\
 

\noindent \textbf{Пример команды:} ::\mbox{ mg\_vsp }($3$, $11$, $12$, $13$, $3$, $24$, $25$, $26$);

\subsection{Найти длину вектора}

\noindent \textbf{Команда для почты:} \mbox{ mg\_vl }

\noindent \textbf{Текст задания:} Найти длину вектора $vec{v}$

\noindent \textbf{Аргументы команды:}\\ $v$ --- последовательность значений, первое из которых есть целое число, определяющее количество следующих за ним значений; каждое это  значение должно быть действительным числом;\\
 

\noindent \textbf{Пример команды:} ::\mbox{ mg\_vl }($3$, $10$, $11$, $12$);

\subsection{Найти математическое ожидание и дисперсию непрерывной случайной величины, заданной функцией плотности вероятности}

\noindent \textbf{Команда для почты:} \mbox{ mc\_crvdfdev }

\noindent \textbf{Текст задания:} Найти математическое ожидание и дисперсию непрерывной случайной величины, заданной функцией плотности вероятности $f(\x)$ на промежутке $a<x<b$ (равной нулю в остальных точках).

\noindent \textbf{Аргументы команды:}\\ $f(\x)$ ---  Значение должно быть функцией; Значение должно быть выражением, содержашим только x в качестве переменной;\\
$a$ ---  Значение должно быть действительным числом;\\
$b$ ---  Значение должно быть действительным числом;\\
 

\noindent \textbf{Пример команды:} ::\mbox{ mc\_crvdfdev }($x$, $0$, $1$);

\subsection{Найти математическое ожидание непрерывной случайной величины, заданной функцией плотности вероятности}

\noindent \textbf{Команда для почты:} \mbox{ mc\_crvdfev }

\noindent \textbf{Текст задания:} Найти математическое ожидание непрерывной случайной величины, заданной функцией плотности вероятности $f(\x)$ на промежутке $a<x<b$ (равной нулю в остальных точках).

\noindent \textbf{Аргументы команды:}\\ $f(\x)$ ---  Значение должно быть функцией; Значение должно быть выражением, содержашим только x в качестве переменной;\\
$a$ ---  Значение должно быть действительным числом;\\
$b$ ---  Значение должно быть действительным числом;\\
 

\noindent \textbf{Пример команды:} ::\mbox{ mc\_crvdfev }($x$, $0$, $1$);

\subsection{Найти определенный интеграл}

\noindent \textbf{Команда для почты:} \mbox{ mm\_dint }

\noindent \textbf{Текст задания:} Найти определенный интеграл $\int^b_a f(\x)d\x$.

\noindent \textbf{Аргументы команды:}\\ $f(\x)$ ---  Значение должно быть функцией; Значение должно быть выражением, содержашим только x в качестве переменной;\\
$a$ ---  Значение должно быть действительным числом;\\
$b$ ---  Значение должно быть действительным числом;\\
 

\noindent \textbf{Пример команды:} ::\mbox{ mm\_dint }($sin(x)+x$, $0$, $1$);

\subsection{Найти неопределенный интеграл}

\noindent \textbf{Команда для почты:} \mbox{ mm\_iint }

\noindent \textbf{Текст задания:} Найти неопределенный интеграл $\int f(\x)d\x$.

\noindent \textbf{Аргументы команды:}\\ $f(\x)$ ---  Значение должно быть функцией; Значение должно быть выражением, содержашим только x в качестве переменной;\\
 

\noindent \textbf{Пример команды:} ::\mbox{ mm\_iint }($sin(x)+x$);

\subsection{Решение системы линейных уравнений 3 на 3 методом Крамера}

\noindent \textbf{Команда для почты:} \mbox{ ma\_slesk33 }

\noindent \textbf{Текст задания:} Решить систему линейных уравнений методом Крамера $$\left\{\begin{array}{l}a_{11}\cdot x_{1}+a_{12}\cdot x_{2}+a_{13}\cdot x_{3}=b_{1},\\a_{21}\cdot x_{1}+a_{22}\cdot x_{2}+a_{23}\cdot x_{3}=b_2,\\a_{31}\cdot x_{1}+a_{32}\cdot x_{2}+a_{33}\cdot x_{3}=b_{3}.\end{array}\right.$$

\noindent \textbf{Аргументы команды:}\\ $a_{11}$ ---  Значение должно быть действительным числом;\\
$a_{12}$ ---  Значение должно быть действительным числом;\\
$a_{13}$ ---  Значение должно быть действительным числом;\\
$a_{21}$ ---  Значение должно быть действительным числом;\\
$a_{22}$ ---  Значение должно быть действительным числом;\\
$a_{23}$ ---  Значение должно быть действительным числом;\\
$a_{31}$ ---  Значение должно быть действительным числом;\\
$a_{32}$ ---  Значение должно быть действительным числом;\\
$a_{33}$ ---  Значение должно быть действительным числом;\\
$b_{1}$ ---  Значение должно быть действительным числом;\\
$b_{2}$ ---  Значение должно быть действительным числом;\\
$b_{3}$ ---  Значение должно быть действительным числом;\\
 

\noindent \textbf{Пример команды:} ::\mbox{ ma\_slesk33 }($11$, $12$, $13$, $21$, $22$, $23$, $31$, $32$, $33$, $-1$, $-2$, $-3$);

\subsection{Найти определитель матрицы 3 на 3}

\noindent \textbf{Команда для почты:} \mbox{ ma\_mdet33 }

\noindent \textbf{Текст задания:} Найти определитель матрицы $\begin{pmatrix} a_{11} & a_{12} & a_{13} \\ a_{21} & a_{22} & a_{23} \\ a_{31} & a_{32} & a_{33} \end{pmatrix}$

\noindent \textbf{Аргументы команды:}\\ $a_{11}$ ---  Значение должно быть действительным числом;\\
$a_{12}$ ---  Значение должно быть действительным числом;\\
$a_{13}$ ---  Значение должно быть действительным числом;\\
$a_{21}$ ---  Значение должно быть действительным числом;\\
$a_{22}$ ---  Значение должно быть действительным числом;\\
$a_{23}$ ---  Значение должно быть действительным числом;\\
$a_{31}$ ---  Значение должно быть действительным числом;\\
$a_{32}$ ---  Значение должно быть действительным числом;\\
$a_{33}$ ---  Значение должно быть действительным числом;\\
 

\noindent \textbf{Пример команды:} ::\mbox{ ma\_mdet33 }($1$, $2$, $3$, $4$, $5$, $6$, $7$, $8$, $9$);

\subsection{Найти обратную матрицу 3 на 3}

\noindent \textbf{Команда для почты:} \mbox{ ma\_minv33 }

\noindent \textbf{Текст задания:} Найти обратную матрицу для матрицы $\begin{pmatrix} a_{11} & a_{12} & a_{13} \\ a_{21} & a_{22} & a_{23} \\ a_{31} & a_{32} & a_{33} \end{pmatrix}$

\noindent \textbf{Аргументы команды:}\\ $a_{11}$ ---  Значение должно быть действительным числом;\\
$a_{12}$ ---  Значение должно быть действительным числом;\\
$a_{13}$ ---  Значение должно быть действительным числом;\\
$a_{21}$ ---  Значение должно быть действительным числом;\\
$a_{22}$ ---  Значение должно быть действительным числом;\\
$a_{23}$ ---  Значение должно быть действительным числом;\\
$a_{31}$ ---  Значение должно быть действительным числом;\\
$a_{32}$ ---  Значение должно быть действительным числом;\\
$a_{33}$ ---  Значение должно быть действительным числом;\\
 

\noindent \textbf{Пример команды:} ::\mbox{ ma\_minv33 }($1$, $2$, $3$, $4$, $5$, $6$, $7$, $8$, $9$);

\subsection{Решение квадратного уравнения}

\noindent \textbf{Команда для почты:} \mbox{ ma\_sqe }

\noindent \textbf{Текст задания:} Решить квадратное уравнение $a\cdot x^2+b\cdot x+c=0$.

\noindent \textbf{Аргументы команды:}\\ $a$ ---  Значение должно быть действительным числом; Значение должно быть не равно нулю;\\
$b$ ---  Значение должно быть действительным числом;\\
$c$ ---  Значение должно быть действительным числом;\\
 

\noindent \textbf{Пример команды:} ::\mbox{ ma\_sqe }($1$, $0$, $0$);

